\section{Yhteenveto}

Tässä tutkielmassa keskityttiin pistepilvien visualisointitekniikoihin erityisesti laitossuunnitteluohjelmiston tarpeisiin. Pistepilvet ovat laserkeilaimilla mitattuja, suuria pistejoukkoja, jotka kuvastavat mitattavan kohteen pinnanmuotoja. Pistepilviä käytetään muun muassa arkeologiassa, rakentamisessa ja erilaisissa suunnittelutehtävissä. Laitossuunnittelussa pistepilviä käytetään useimmiten muutostöiden yhteydessä. Laserkeilauksella saadaan kohteesta tuotettua kustannustehokkaasti ajantasalla oleva tarkka 3d-malli, joka on välttämätön suurempien muutostöiden onnistumisen kannalta.   

%Ennen visualisointia pistepilviä täytyy esikäsitellä. Eri kohdista suoritetut laserkeilaukset täytyy sovittaa yhteiseen koordinaatistoon rekisteröinniksi kutsutussa työvaiheessa. Joissakin sovelluksissa on tarpeen muodostaa pistepilvestä kolmioverkko, joka voidaan varjostaa muiden 3d-mallin objektien tapaan. Pilvestä voidaan myös yrittää automaattisesti löytää joitakin primitiivejä, kuten tasoja tai sylinterejä. Kompakti esitys pistepilvestä saadaan luomalla panoramakuvia projisoimalla pisteitä laserkeilaimen ympäriltä kuvatasolle. 

Laitossuunnitteluohjelmistossa käytetyt pistepilvet kattavat usein laajoja alueita ja sisältävät niin paljon pisteitä, ettei kaikkia ehditä piirtämään ruudulle jokaisella ruudunpäivityksellä, eikä kokonaisia pistepilviä haluta pitää keskusmuistissa tai näytönohjaimen muistissa. Näihin haasteisiin on vastattu hierarkisilla tietorakenteilla, jotka mahdollistavat pistepilven asteittaisen lataamisen ja visualisoinnin eri tarkkuuksilla. 

Luvussa \ref{kirjallisuus} esiteltiin tietorakenteita, joista useat perustuivat avaruutta pienempiin osiin jakaviin puihin, joiden sisäsolmut näytteistävät pistepilveä eri tarkkuuksilla. Qsplat \cite{qsplat} käytti pistejoukkojen rajauspalloista koostuvaa puuta joka visualisoitiin rajauspallojen kokoisilla täplillä. Samaa ajatusta jatkettiin peräkkäispuissa \cite{spt}\cite{ip}, jotka hyödynsivät lisäksi näytönohjaimen laskentatehoa. Sisäkkäispistepuut \cite{scheiblauer}\cite{potree} mahdollistivat pistepilven asteittaisen tarkentamisen ja jopa pistedatan lataamisen verkon yli. Peräkkäis- ja sisäkkäispistepuut perustuvat avaruuden jakamiseen samankokoisiin palasiin jokaisella puun tasolla. Vaihtoehtoinen lähestymistapa on käyttää kd-puuta \cite{richter}\cite{smooth}, jonka vahvuutena on puun tasapainoisuus. Näytönohjainten muistien kasvaessa myös ei-hierarkiset tekniikat \cite{clod}\cite{progressive} alkavat olla mahdollisia isojenkin pistepilvien visualisoinnissa. 

Lähempään tarkasteluun valittiin sisäkkäispistepuut, sillä niiden suorituskyky on todettavissa valmiista toteutuksesta\footnote{www.potree.org/} ja ne vaikuttavat vastaavan luvussa \ref{usecase} esitettyihin vaatimuksiin. Sisäkkäispistepuut mahdollistavat eri tarkkuustasojen visualisoinnin käytettävissä olevasta piirtoajasta ja laitteistosta riippuen. Sisäkkäispistepuut mahdollistavat myös pistepilven asteittaisen tarkentamisen monen ruudunpäivityksen ajan, mikä sopii laitossuunnitteluohjelmiston tarpeisiin. Usean laserkeilauksen sovittaminen samaan sisäkkäispistepuuhun mahdollistaa pisteiden harventamisen ja globaalin enimmäistiheyden. 

Sisäkkäispistepuiden rakenne mahdollistaa luvussa \ref{kompressio} esitetyn yksinkertaisen kompressiotekniikan, jossa pisteiden absoluuttiset koordinaatit vaihdetaan indekseihin solmuissa sijaitseviin ruudukoihin. Suhteellisten koordinaattien käyttäminen nopeuttaa lataamisen ja tallentamisen lisäksi myös joitakin laitossuunnitteluohjelmistossa tarvittavia operaatioita, kuten pistepilvien transformaatioita. Myös pienentynyt tallennustilan tarve on huomattava etu laitossuunnitteluohjelmistossa.
 
Luvussa \ref{render} esiteltiin yksinkertainen visualisointialgoritmi sisäkkäispistepuille, joka pitää jokaisella ruudunpäivityksellä piirrettävää karkeaa yleiskuvaa pistepilvestä näytönohjaimen muistissa. Tämän jälkeen loput pisteet voidaan piirtää niiden tärkeysjärjestyksessä, eli oktettipuun solmujen ruudulle projisoidun koon mukaan. Luvussa \ref{tulokset} selvisi, että esitetty visualisointialgoritmi on suoraviivaista verrokkia hitaampi, mutta visuaalinen tulos on parempi samalla määrällä piirrettyjä pisteitä.    

Joihinkin operaatioihin tutkielmassa esitellyn kaltaiset sisäkkäispistepuut eivät kuitenkaan ole optimaalisia. Kun halutaan tarkastella tietyn alueen sisältämiä pisteitä, täytyy niitä etsiä sisäkkäispistepuusta koko polulta juuresta lehtiin. Tämä hidastaa esimerkiksi yksittäisten pisteiden tai pistejoukkojen valitsemista kursorilla. Käytännössä suorituskyky ei kärsinyt liikaa, joten voidaan todeta sisäkkäispistepuiden soveltuvan hyvin pistepilvien visualisointiin laitossuunnitteluohjelmistoissa.