\section{Yhteenveto}

Tässä tutkielmassa keskityttiin pistepilvien visualisointitekniikoihin erityisesti laitossuunnitteluohjelmiston tarpeisiin. laitossuunnitteluohjelmistossa käytetyt pistepilvet kattavat usein laajoja alueita ja sisältävät niin paljon pisteitä, ettei kaikkia ehditä piirtämään ruudulle jokaisella ruudunpäivityksellä, eikä kokonaisia pistepilviä haluta pitää keskusmuistissa tai näytönohjaimen muistissa. Näihin haasteisiin on vastattu hierarkisilla tietorakenteilla, jotka mahdollistavat pistepilven asteittaisen lataamisen ja visualisoinnin eri tarkkuuksilla. 

Luvussa \ref{kirjallisuus} esiteltiin tietorakenteita, joista useat perustuivat avaruutta pienempiin osiin jakaviin puihin, joiden sisäsolmut näytteistävät pistepilveä eri tarkkuuksilla. Rusinkiewiczin ja Levoyn Qsplat \cite{qsplat} käytti pistejoukkojen rajauspalloista koostuvaa puuta joka visualisoitiin rajauspallojen kokoisilla täplillä. Samaa ajatusta jatkettiin peräkkäispuissa \cite{spt}\cite{ip}, jotka hyödynsivät lisäksi näytönohjaimen laskentatehoa. Sisäkkäispistepuut \cite{scheiblauer}\cite{potree} mahdollistivat pistepilven asteittaisen tarkentamisen ja jopa pistedatan lataamisen verkon yli. Peräkkäis- ja sisäkkäispistepuut perustuvat avaruuden jakamiseen samankokoisiin palasiin jokaisella puun tasolla. Vaihtoehtoinen lähestymistapa on käyttää kd-puuta \cite{richter}\cite{smooth}, jonka vahvuutena on puun tasapainoisuus. Näytönohjainten muistien kasvaessa myös ei-hierarkiset tekniikat \cite{clod}\cite{progressive} alkavat olla mahdollisia isojenkin pistepilvien visualisoinnissa. 

Lähempään tarkasteluun valittiin sisäkkäispistepuut, sillä niiden suorituskyky on todettavissa valmiista toteutuksesta\footnote{www.potree.org/} ja ne vaikuttavat vastaavan luvussa \ref{usecase} esitettyihin vaatimuksiin. Lisäksi sisäkkäispistepuiden rakenne mahdollistaa luvussa \ref{kompressio} esitetyn yksinkertaisen kompressiotekniikan, jossa pisteiden absoluuttiset koordinaatit vaihdetaan indekseihin solmuissa sijaitseviin ruudukoihin. Suhteellisten koordinaattien käyttäminen nopeuttaa lataamisen ja tallentamisen lisäksi myös joitakin laitossuunnitteluohjelmistossa tarvittavia operaatioita, kuten pistepilvien transformaatioita. 

Sisäkkäispistepuut mahdollistavat pistepilven asteittaisen tarkentamisen monen ruudunpäivityksen ajan, mikä sopii laitossuunnitteluohjelmiston tarpeisiin. Joihinkin operaatioihin se ei kuitenkaan ole optimaalinen. Kun halutaan tarkastella tietyn alueen sisältämiä pisteitä, täytyy niitä etsiä sisäkkäispistepuusta koko polulta juuresta lehtiin. Tämä hidastaa esimerkiksi yksittäisten pisteiden tai pistejoukkojen valitsemista kursorilla.

