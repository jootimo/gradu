% Tässä metadata gradujen pdf/a tiedostomuotoa varten. Muista päivittää! Tähän ei voi laittaa ääkkösiä
\begin{filecontents*}{\jobname.xmpdata}
\Title{Pistepilvien visualisointi laitossuunnitteluohjelmistossa}
\Author{Timo Heinonen}
\end{filecontents*}
\documentclass{wihuri}
%\usepackage{isolatin1} % Saadaan ääkköset toimimaan !
%\usepackage[latin1]{inputenc} % Saadaan oikeat merkit
\usepackage[utf8]{inputenc} % Tällä toimii utf-8
\usepackage[T1]{fontenc}      % Ja tämä liittyy edelliseen
\usepackage[finnish]{babel} %Suomenkielinen tavutus
\usepackage{tytiivis} %Tiivistelmäsivun laatimiseksi
\usepackage{graphicx}%Saadaan kuvat toimimaan
\usepackage{lastpage}
\usepackage{subfiles}
\usepackage{url}
\usepackage[dvipsnames]{xcolor}
\usepackage{tikz}
\usepackage{tikz-3dplot}
\usetikzlibrary{positioning, intersections,decorations.pathreplacing,positioning}
\usepackage{forest}
\usepackage[a-1b]{pdfx}  % pdf:n tulee graduissa olla pdf/a-1b-standardin mukaista. Tällä ja pdflatexia käyttämällä se onnistuu
\usepackage[pdfa]{hyperref} % Tämä tarvitaan, jos haluaa sisällysluettelon klikattavaan muotoon. 
\hypersetup{
    colorlinks = false,
    linkbordercolor = {white}
}
\usepackage{times}
\usepackage{amsmath}

% Algoritmipaketti ja pari hyödyllistä komentoa
\usepackage[ruled,linesnumbered]{algorithm2e}
\renewcommand*{\listalgorithmcfname}{Lista algoritmeista}
\renewcommand*{\algorithmcfname}{Algoritmi}
\renewcommand*{\algorithmautorefname}{Algoritmi}
\newcommand{\perusalgoritmi}[0]{\SetKwData{Left}{left}\SetKwData{This}{this}\SetKwData{Up}{up}\SetKwFunction{Union}{Union}\SetKwFunction{FindCompress}{FindCompress}\SetKwInOut{Input}{Syöte}\SetKwInOut{Output}{Tuloste}\BlankLine}
\definecolor{kommentticolor}{RGB}{30,90,65}
\newcommand{\kommentti}[1]{\textcolor{kommentticolor}{\footnotesize\tcp{#1}}\normalsize}


% pdflatex vaatii, että kuvat ovat jotain muuta kuin eps-muotoisia. Esimerkiksi pdf käy mainiosti vektorikuville 
% ja png pikselikuville. Pikselikuvat mennevät paremmin läpi gradujen validoinnista.
%
% Esimerkkejä uusien käskyjen määrittelyistä.
% Käsky \mathbi{``vektorin symboli''} luo boldin italicin kirjaimen. Kreikkalaisille
% kirjaimille taitaa olla pakko käyttää \pmb:tä.
\newcommand{\mathbi}[1]{\textbf{\em #1}}
% Käsky \der luo derivaatan d:n
\newcommand{\der}{\mbox{d}}
%

% Englanninkielinen termi
\newcommand{\engl}[1]{(engl. \emph{#1})}

% Väliotsikko ilman numeroa. Ei tyhjää riviä tämän jälkeen.
\newcommand{\subtitle}[1]{\vspace{0.5cm}\noindent\large\textbf{#1}\normalsize\\\noindent}

% Koodia
\newcommand{\code}[1]{\noindent\footnotesize\texttt{#1}\normalsize}
\newcommand{\tab}[0]{\hspace*{1cm}}

\begin{document}
\title{Pistepilvien visualisointi laitossuunnitteluohjelmistossa}
\opinnayte{Pro Gradu}
\author{Timo Heinonen}
\vuosi{2018}
\oppiaine{Tietojenkäsittelytiede}
\tarkastaja{P.P.}
\tarkastaja{H.H.}
\maketitle
\newpage
\thispagestyle{empty}
\vspace*{10cm}

\vfill

\hspace*{-2cm}\parbox{\textwidth}{Turun yliopiston laatujärjestelmän mukaisesti
  tämän julkaisun alkuperäisyys on tarkastettu Turnitin
  OriginalityCheck-järjestelmällä} 
%Huomaa, että joudut kuitenkin printtaamaan tämän sivun erikseen
%kaksipuoleseksi kannen kanssa.


\newpage
\begin{tiivistelma}%
        {Tulevaisuuden teknologioiden laitos}%
        {Timo Heinonen}%
        {Pistepilvien visualisointi laitossuunnitteluohjelmistossa}
        {Pro Gradu, \pageref{LastPage} s., 3 liites.}%
        {Tietojenkäsittelytiede}% Oppiaine
        {\today}%
	Juuh elikkäs gradutyötä...
\end{tiivistelma}

\tableofcontents %Sisällysluettelo
\newpage

\subfile{johdanto.tex}

%\section*{Johdanto}
%\addcontentsline{toc}{section}{Johdanto}
%Näin tehtynä Johdannolle ei tule numeroa sisälllysluetteloon

\subfile{tietorakenteet.tex}


%\subsection{Pistedatan pakkaaminen}
%Octreen pisteen koordinaatin ilmaiseminen kahdella tavulla: Tavun kumpikin bitti ilmaisee $s/2^{16}$ etäisyyttä suorakulmaisen särmiön vasemmasta alanurkasta. $s$ on särmiön sivun pituus. \cite{Elseberg}


\subfile{mun_tietorakenne.tex}


\section{Tietorakenteen arviointi}\label{arviointi}

\subsection{Tietorakenteen rakentaminen ja lataaminen}
\subsection{Visualisointi}
\subsection{Pisteiden valitseminen}
\subsection{jotain muuta?}
\subsection{jotain muuta?}

%Memory overhead M ~ 1/(alpha-1), alpha=branching factor: D UGUET F., D RETTAKIS G.: Flexible point-based rendering on mobile devices. Computer Graphics and Applications 24(4) (2004), 57–63.


\section{Johtopäätökset}









%\newpage
% Rivinväli pienemmäksi viiteluettelossa. Fonttia on vaihdettava, jotta käsky
% toimisi !
\renewcommand{\baselinestretch}{1}\large\normalsize


\addcontentsline{toc}{section}{Viitteet}

\clearpage
%\bibliographystyle{plain} % We choose the "plain" reference style
\bibliographystyle{wihuri}
\bibliography{refs} % Entries are in the "refs.bib" file
%
%\begin{thebibliography}{50}% Viiteluettelo. TÄTÄ ON PAKKO KÄYTTÄÄ !
% Jaa, ai miksi ? No, koska tällä tavalla se on vaan niin pirusti
% helpompaa.
%\bibitem{lshort} T. Oetiker, H. Partl, I. Hyna and E. Schlegl,
%Not so short introduction to \LaTeX 2e, 1998
%\end{thebibliography}
%
% Vaihtoehtoisesti thebibliography ympäristölle voi käyttää BibTeX
% tietokantaa, jonka voit luoda tai käyttää olemassaolevaa (esim.
% Wihurilla). Suosittelemme tätä lämpimästi!
%
% Bibtex-tietokannan saa helposti tehtyä esim TeXMakerilla. Sitten
% vaan ajetaan latex gradu, bibtex gradu, latex gradu ja latex
% gradu. Ja TADAA viitteet ovat oikeassa järjestyksessä.
%
%\bibliography{/var/bib/yhdistetty}
%\bibliographystyle{wihuri}
%
\end{document} % Se oli siinä !
